V této práci se soustředíme na vylepšení návrhu task-oriented dialogových systémů v době rychle se vyvíjejících technologií umělé inteligence.
Je žádoucí zajistit, aby systémy používající dialogové rozhraní byly snadněji adaptovatelné a rozšiřitelné.
Námi navržené techniky řeší oba tyto problémy a mohou výrazně snížit cenu vývoje i nasazení aplikací dialogových systémů.


Firstly, we present a weakly supervised automatic data annotation pipeline that can transform raw data into a refined set of semantically coherent concepts, bypassing the need for exhaustive manual annotations and significantly streamlining the development process.

We also explore the largely uninvestigated field of latent variable models in task-oriented dialogue system modeling.
These models offer excellent capabilities with the potential to uncover the structure of behavioral patterns seen in the dialogue through inspection of the latent space and comparison with actions taken by the model.

Moreover, we harness the power of large pre-trained language models using in-context learning, following progress in the field.
Our proposed retrieval-augmented method performs well with merely a few training examples.
It shows great promise regarding human evaluation, implying a substantial leap in efficiently using computational resources to train conversational AI.
This brings us closer to more flexible and general-purpose systems.

In summary, the research presented in this thesis brings exciting innovations, uncovering new pathways in developing task-oriented dialogue systems.
It moves towards a future where dialogue systems can adapt, learn, and interact efficiently, potentially transforming how people interact with technology.