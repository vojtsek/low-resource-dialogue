% %%%%%%%%%%%%%%%%%%%%%%%%%%%%%%%%%%%%%%%%%%%%%%%%%%%%%%%%%%%%%%%%%%%%%%%%%%%%
\chapter{Introduction}%
\label{chap:intro}
Human language is a convenient and the most natural means of communication for human beings.
It is therefore desirable to implement an interface that mimics natural language and allows humans to interact with computers in the same way as they would with other human individuals.

To achieve this goal, we need to be able to transfer information between human users and the computer.
Humans most often use speech or written text to encode and transfer information and there are techniques that deal with this kind of encoding such as Automatic Speech Recognition (ASR), Optical Character Recognition (OCR), and Text-to-speech Synthesis (TTS).
However, to perform a meaningful dialogue, we need more than just to mimic the interface.
The computer should be able to understand the meaning of utterances in the context and provide relevant responses.
In this work, we focus on this part of the problem, i.e. we do not care about the process of encoding or decoding natural language in a signal such as speech.
Rather we assume textual interfaces for both input and output.
Put simply, the task of a Dialogue System (DS) is to generate the correct natural language response \textit{r} given the natural language user utterance \textit{u} and context \textit{c}.
The dialogue is a \textit{turn-taking} conversation, i.e. participants (user and system) communicate in alternating \textit{turns}.

The ultimate goal is to construct a dialogue agent that provides meaningful responses to all kinds of questions taking the conversation history into account.
Such agent would effectively pass the Turing test, the holy grail of sort for the field of Artificial Intelligence.
This goal is most likely far from being achieved, nevertheless, in many real life cases we don't need such complexity.

Dialogue systems promise a convenient means of communication between human and computers.
They allow voice interaction, making it especially well suited for applications that should not disrupt attention such as car control.
Systems capable of human-like conversation and accomplishing given task have huge potential to automate tech support processes, call centers or serve as personal assistants.

Despite some successful dialogue system deployments, dialogue systems still suffer from a number of drawbacks.
Usually, the DSs are tailored to specific applications and it is hard to apply them in other domains.
This results in bad scalability and inflexible applications.
Another problem is that they require a lot of annotation for the training data, especially in the task-oriented setting.
Last but not least, there seems to be a trade-off between interpretability and performance or scalability of the systems in the case of neural network based models.

In this thesis, we aim to propose solutions to some of these problems, especially in the task-oriented setting.
We now outline the main goals we want to achieve:
\begin{itemize}
    \item Make the task-oriented dialogue systems more scalable and easier to extend.
    \item Utilize transfer learning so that domain adaptation is easier.
    \item Enable the dialogue systems to leverage large unannotated data sets and consequently train more robust models.
\end{itemize}
Scalability and domain adaptation goes hand in hand.
We focus on reduction of the amount of annotation needed to train a system and on knowledge abstraction in order to make transfer learning possible.
To be able to leverage larger data sets, we explore unsupervised techniques which do not require annotation and therefore make the data collection process substantially easier.

% %%%%%%%%%%%%%%%%%%%%%%%%%%%%%%%%%%%%%%%%%%%%%%%%%%%%%%%%%%%%%%%%%%%%%%%%%%%%


